
\documentclass[a4paper,10pt]{article}
\author{\textbf{Choi Pooreunhaneul}}
\title{\textbf{2nd sym. Theorems about Number Thoery}}
\usepackage{kotex}

% 이거 문서 여백 조절이니 알잘딱하게 맞춰쓰세요
\usepackage[scale=0.75,twoside,bindingoffset=10mm,a4paper]{geometry}
\usepackage{amsmath,amsfonts,amssymb,mathrsfs,mathtools}
\usepackage{adjustbox,float,graphicx,relsize}
\usepackage{xcolor}
\usepackage{tikz}
\usepackage{setspace}
\usepackage{expl3}
\usepackage{xparse}
\usepackage{enumitem}
\usepackage{forloop}
\begin{document}

% generates the title
\maketitle

% insert the table of contents  
\section{The Quadratic Reciprocity Law}
\textbf{Definition.} Let \textit{p} be an odd prime and \textit{gcd(a,p)=1}. If the quadratic congruence \textit{$x^{2}\equiv a \; (mod \; p)$} has a sol, then \textit{a} is said to be a quadratic residue of \textit{p}. Otherwise, \textit{a} is called a quadratic nonresidue of \textit{p}. \\ \\
\textbf{Theorem 1.1 Euler's criterion.} Let \textit{p} be an odd prime and \textit{gcd(a,p)=1}. Then \textit{a} is a quadratic residue of \textit{p} iff \textit{$a^{(p-1)/2}\equiv1 \; (mod \; p)$}. \\ \\
\textbf{Corollary.} Let \textit{p} be an odd prime and \textit{gcd(a,p)=1}. Then \textit{a} is a quadratic residue or nonresidue of \textit{p} according to whether
\begin{align}
a^{(p-1)/2}\equiv1 \; (mod \; p) \qquad \textnormal{or} \qquad a^{(p-1)/2}\equiv-1 \; (mod \; p)
\end{align}
\textbf{Definition.} Let \textit{p} be an odd prime and let \textit{gcd(a,p)=1}. The Legendre symbol \textit{(a/p)} is defined by
\begin{align}
1 \; \textnormal{if} \; a \; \textnormal{is a quadratic residue of} \; p \\
-1 \; \textnormal{if} \; a \; \textnormal{is a quadratic nonresidue of} \; p 
\end{align}
\textbf{Theorem 1.2.} Let \textit{p} be an odd prime and let \textit{a} and \textit{b} be int. that are relatively prime to \textit{p}. Then the Legendre symbol has the following properties:
\begin{align}
&(a) \; \textnormal{If} \; a\equiv b \; (mod \; p), \; \textnormal{then} \; (a/p)=(b/p). \\
&(b) \; (a^{2}/p)=1 \\ 
&(c) \; (a/p)=a^{(p-1)/2} \; (mod \; p) \\
&(d) \; (ab/p)=(a/p)(b/p) \\
&(e) \; (1/p) \; \textnormal{and} \; (-1/p)=(-1)^{(p-1)/2}
\end{align}
\textbf{Corollary.} If \textit{p} is an odd prime, then
\begin{align}
(-1/p) = \left \{ \begin{array}{cc} 1 \quad \textnormal{if} \; p \equiv 1 \; (mod \; 4) \\ 2 \quad \textnormal{if} \; p \equiv 3 \; (mod \; 4)\end{array} \right.
\end{align}
\textbf{Theorem 1.3.} If \textit{p} is an odd prime, then
\begin{align}
\sum^{p-1}_{a=1}(a/p) = 0
\end{align}
\textbf{Corollary.} The quadratic residues of an odd prime \textit{p} are congruent modulo \textit{p} to the even powers of a primitive root \textit{r} of \textit{p}; the quadratic nonresidues are congruent to the odd powers of \textit{r}. \\ \\
\textbf{Theorem 1.4 Gauss' lemma.} Let \textit{p} be an odd prime and let gcd(\textit{a,p}) = 1. If \textit{n} denotes the number of int. in the set
\begin{align}
S = \left\{ a, 2a, 3a, \ldots, \left( \cfrac{p-1}{2} \right)a \right\}
\end{align}
whose remainders upon division by \textit{p} exceed \textit{p/2}, then
\begin{align}
(a/p) = (-1)^{n}
\end{align}
\textbf{Theorem 1.5.} If \textit{p} is an odd prime, then
\begin{align}
(2/p) = \left \{ \begin{array}{cc} 1 \quad \textnormal{if} \; p \equiv \pm 1 \; (mod \; 8) \\ -1 \quad \textnormal{if} \; p \equiv \pm 3 \; (mod \; 8)\end{array} \right.
\end{align}
\textbf{Corollary.} If \textit{p} is an odd prime, then
\begin{align}
(2/p) = (-1)^{(p^{2}-1)/8}
\end{align}
\textbf{Theorem 1.6.} If \textit{p} and $2p+1$ are both odd primes, then the int. $2(-1)^{(p^{2}-1)/8}$ is a primitive root of \textit{2p+1}. \\ \\
\textbf{Lemma.} If \textit{p} is an odd prime and \textit{a} an odd int, with gcd(\textit{a,p}) = 1, then
\begin{align}
(a/p) = (-1)^{\sum^{(p-1)/2}_{k=1}[ka/p]}
\end{align}
\textbf{Theorem 1.7 Quadratic Reciprocity Law.} If \textit{p} and \textit{q} are distinct odd primes, then
\begin{align}
(p/q)(q/p) = (-1)^{\frac{p-1}{2}\frac{q-1}{2}}
\end{align}
\textbf{Corollary 1.} If \textit{p} and \textit{q} are distinct odd primes, then 
\begin{align}
(p/q)(q/p) = \left \{ \begin{array}{cc} 1 \quad \textnormal{if} \; p \equiv 1 \; or \; q \equiv 1 \; (mod \; 4) \\ -1 \quad \textnormal{if} \; p \equiv q \equiv 3 \; (mod \; 4)\end{array} \right.
\end{align}
\textbf{Corollary 2.} If \textit{p} and \textit{q} are distinct odd primes, then 
\begin{align}
(p/q) = \left \{ \begin{array}{cc} (q/p) \quad \textnormal{if} \; p \equiv 1 \; or \; q \equiv 1 \; (mod \; 4) \\ -(q/p) \quad \textnormal{if} \; p \equiv q \equiv 3 \; (mod \; 4)\end{array} \right.
\end{align}
\textbf{Theorem 1.8.} If \textit{p} $\neq$ 3 is an odd prime, then
\begin{align}
(3/q) = \left \{ \begin{array}{cc} 1 \quad \textnormal{if} \; p \equiv \pm 1 \; (mod \; 12) \\ -1 \quad \textnormal{if} \; p \equiv \pm 5 \; (mod \; 12)\end{array} \right.
\end{align}
\textbf{Theorem 1.9.} If \textit{p} is an odd prime and gcd(\textit{a,p}) = 1, then the congruence
\begin{align}
x^{2} \equiv a \; (mod \; p^{n}) \quad n \geq 1 
\end{align}
has a sol. iff (\textit{a/p}) = 1. \\ \\ 
\textbf{Theorem 1.10.} Let \textit{a} be an odd int. Then we have the following:
\begin{align}
&(a) \; x^{2} \equiv a \; (mod \; 2) \; \textnormal{always has a sol.} \\
&(b) \; x^{2} \equiv a \; (mod \; 4) \; \textnormal{has a sol. iff} \; a \equiv 1 \; (mod \; 4) \\
&(c) \; x^{2} \equiv a \; (mod \; 2^{n}), \textnormal{for} \; n\geq 3, \textnormal{has a sol. iff} \; a \equiv 1 \; (mod \; 8)
\end{align} \
\newpage
\textbf{Theorem 1.11.} Let $n = 2^{k_{0}}p_{1}^{k_{1}}\cdots p_{r}^{k_{r}}$ be the prime factorization of \textit{n} $>$ 1 and let $\textnormal{gcd}(\textit{a,n}) = 1$. Then $x^{2} \equiv a \; (mod \; n)$ is solvable iff.
\begin{align}
&(a) \; (a/p_{i}) = 1 \; \textnormal{for} \; i = 1, 2, \ldots, r; \\ 
&(b) \; a \equiv 1 \; (mod \; 4) \; \textnormal{if} \; 4|n, \; \textnormal{but} \; 8 \nmid n; \; a \equiv 1 \; (mod \; 8) \; \textnormal{if} \; 8|n.
\end{align}
\textbf{Definition. Jacobi Symbol.} Defined as
\begin{align}
(a/p) = \left \{ \begin{array}{cc} 0 \quad p|a \\ 1 \quad p\nmid a \; & \; \textnormal{residue} \\ -1 \; p\nmid a \; & \; \textnormal{nonresidue} \end{array} \right.
\end{align}
\textbf{Theorem 1.12.} For odd positive int. \textit{$b, \;b_{1}, \; b_{2}$} and \textit{$a, \;a_{1}, \; a_{2}$},
\begin{align}
&(a) \; (a/1) = 1 \\
&(b) \; (a_{1}/b)=(a_{2}/b) \; if \; a_{1} \equiv a_{2} \; (mod \; b) \\
&(c) \; (a_{1}a_{2}/b)=(a_{1}/b)(a_{2}/b). \\ 
&(d) \; (a/b_{1}b_{2})=(a/b_{1})(a/b_{2}). 
\end{align}
\textbf{Lemma.} Let int. \textit{r, \; s} is odd. Then,
\begin{align}
&(a) \; \cfrac{rs-1}{2} \equiv \cfrac{r-1}{2}+\cfrac{s-1}{2} \; (mod \; 2) \\
&(b) \; \cfrac{r^{2}s^{2}-1}{8} \equiv \cfrac{r^{2}-1}{8}+\cfrac{s^{2}-1}{8} \; (mod \; 2)
\end{align}
\textbf{Corollary.} Let \textit{$r_{1}, \ldots, r_{m}$} be odd. Then,
\begin{align}
&(a) \; \sum^{m}_{i=1}\cfrac{r_{i}-1}{2} \equiv \cfrac{r_{1}\cdots r_{m}-1}{2} \; (mod \; 2) \\
&(b) \; \sum^{m}_{i=1}\cfrac{r_{i}^{2}-1}{8} \equiv \cfrac{r_{1}^{2}\cdots r_{m}^{2}-1}{8} \; (mod \; 2)
\end{align}
\textbf{Theorem 1.13.} For odd natural num. \textit{a, \; b},
\begin{align}
&(a) \; (-1/b) = (-1)^{\frac{b-1}{2}} \\ 
&(b) \; (2/b) (-1)^{b^{2}-1}{8} \\
&(c) \; (a/b)(b/a) = (-1)^{\frac{a-1}{2}\frac{b-1}{2}}
\end{align}
\textbf{Theorem 1.14.} Let int. \textit{a} not a perfect square. Then $\exists$ $\infty$ly many primes \textit{p} for which \textit{a} is a quad. res. \\ \\
\textbf{Lemma.} Let \textit{a, \; b} natural odd and gcd(\textit{a,b}) = 1. Then,
\begin{align}
&(a) \; \epsilon = \pm 1 \; \Rightarrow \; (\epsilon a/b)(b/a) = (-1)^{\frac{\epsilon a-1}{2}\frac{b-1}{2}} \\
&(b) \; \epsilon_{1}, \; \epsilon_{2} = \pm 1 \; \Rightarrow \; (\epsilon_{1} a/b)(\epsilon_{2} b/a) = (-1)^{\frac{\epsilon_{1} a-1}{2}\frac{\epsilon_{2} b-1}{2}+\frac{\epsilon_{1}-1}{2}\frac{\epsilon_{2}-1}{2}} \\
\end{align}
\newpage
\textbf{Theorem 1.15 Eisenstein's Method.} Let \textit{b} is natural odd and int. \textit{a} is odd. Then, following holds.
\begin{align}
&set. \; a_{1}=a, a_{2}=b, \; a_{i}=2n-1,\; \epsilon_{i} = \pm 1 \\
&a_{n} = q_{n}a_{n+1}+\epsilon_{n}a_{n+2} \; \; with \; \; a_{2} > a_{3} > \cdots > a_{n+2} = 1 \\
&For \;each \; i, \; let. \; s_{i} = \left \{ \begin{array}{cc} 0 \quad \textnormal{if at least one of} \; a_{i+1} \; \textnormal{and} \; \epsilon_{i}a_{i+2} \equiv 1 \; (mod \; 4) \\ 1 \quad \textnormal{if both} \; a_{i+1} \; \textnormal{and} \; \epsilon_{i}a_{i+2} \equiv 3 \; (mod \; 4)\end{array} \right. \\
&let. \; t = \sum^{n}_{i=1}s_{i}. \; \Rightarrow \; (a/b) = (-1)^{t}.
\end{align}
\textbf{Corollary.} Let \; $t = \sum^{n}_{i=1}s_{i}$. Then for any \textit{$k \geq n$},
\begin{align}
(a/b) = (-1)^{t_{k}}\left(\cfrac{a_{k+1}}{a_{k+2}}\right)
\end{align}
\textbf{Theorem 1.16.} The number \textit{N} of sol. with $1\leq x, y \leq p$ \; of \; $y^{2} \equiv ax^{2}+bx+c \; (mod \; p)$ is:
\begin{align}
N = \left \{ \begin{array}{cc} p-(a/p) \quad \textnormal{if} \; p\nmid D  \\ p+(p-1)(a/p) \quad \textnormal{if} \; p \mid D \end{array} \right.
\end{align}
where $D=b^{2}-4ac$. \\ \\
\section{Number of Special Forms}
\textbf{Definition.} If $\sigma(n)=2n$, \textit{n} is perfect number. \\ \\
\textbf{Theorem 2.1.} If $2^{k}-1$ is prime, then $n = 2^{k-1}(2^{k}-1)$ is perfect and every even perfect number is of this form. \\ \\
\textbf{Lemma.} If $a^{k}-1 \; (a>0, k \geq 2)$ is prime, then \textit{a=}2 and \textit{k} is also prime. \\ \\
\textbf{Theorem 2.2.} An even perfect number ends in the digit 6 or 8; equivalently. \\ \\
\textbf{Definition.} $M_{n}=2^{n}-1$ is defined as Mersenne prime. \\ \\
\textbf{Theorem 2.3.} If \textit{p} and $q=2p+1$ are primes, then either $q|M_{p}$ or $q|M_{p}+2$, but not both. \\ \\
\textbf{Theorem 2.4.} If $q=2n+1$ is prime, then we have the following:
\begin{align}
&(a) \; q|M_{n}, \; \textnormal{provided that} \; q \equiv 1 \; (mod \; 8) \; or \; q \equiv 7 \; (mod \; 8) \\
&(b) \; q|M_{n}, \; \textnormal{provided that} \; q \equiv 3 \; (mod \; 8) \; or \; q \equiv 5 \; (mod \; 8) \\
\end{align}
\textbf{Corollary.} If \textit{p} and $q=2p+1$ are both odd primes, with $p\equiv 3 \; (mod \; 4)$, then $q|M_{n}$. \\ \\
\textbf{Theorem 2.5.} If \textit{p} is an odd prime, then any prime divisor of $M_{n}$ is of the form $2kp+1$. \\ \\
\textbf{Theorem 2.6.} If $p$ is an odd prime, then any prime divisor $q$ of $M_{n}$ is of the form $q\equiv \pm1 \; (mod \; 8)$ \\ \\
\textit{\textbf{Remark.}} Define $S_{k}$ by $S_{1}=4$, $S_{k+1}=S_{k}^{2}-2$. \\ Then for prime, $M_{p}$ is prime $\Longleftrightarrow$ $S_{p-1}\equiv 0 \; (mod \; M_{p})$ $\Longleftrightarrow$ $S_{p-2} \equiv \pm 2^{\frac{p+1}{2}} \; (mod \; M_{p})$. \\ \\
\textbf{Theorem 2.7 Euler.} If $n$ is an odd perfect num, then 
\begin{align}
n=p_{1}^{k_{1}}p_{2}^{2j_{2}}\cdots p_{r}^{2j_{r}}
\end{align}
where the $p_{i}$'s are distinct odd primes and $p_{1}\equiv k_{1} \equiv 1 \; (mod \; 4)$. \\ \\
\textbf{Corollary.} If $n$ is an odd perfect, then $n$ is of the form 
\begin{align}
n=p^{k}m^{2}
\end{align}
where $p$ is a prime, $p\nmid m$, and $p\equiv k \equiv 1 \; (mod \; 4)$; in particular, $n \equiv 1 \; (mod \; 4))$. \\ \\
\textbf{Definition.} $m, n$ satisfying $\sigma(m)=\sigma(n)=m+n$ are called amicable numbers. \\ \\
\textit{\textbf{Fact.}} $p=3\cdot 2^{n-1}-1, \; q=3\cdot 2^{n}-1$, and $r=9\cdot 2^{2n-1}-1$ are all primes and $n \geq 2$, then $2^{n}pq$ and $2^{n}r$ are amicable numbers. \\ \\
\textbf{Definition.} $F_{n}=2^{2^{n}}+1$ is called Fermat number. If it is prime, we more specially call it Fermat prime. \\ \\
\textbf{Theorem 2.8.} $F_{5}$ is divisible by 641. \\ \\
\textbf{Theorem 2.9.} $F_{n}$ and $F_{m}$, where $m>n$, gcd($F_{m}, F_{n}$)=1. \\ \\
\textbf{Theorem 2.10 Pepin's test.} For natural $n$, $F_{n}$ is prime iff $3^{\frac{F_{n}-1}{2}}\equiv -1 \; (mod \; F_{n})$. \\ \\
\textbf{Theorem 2.11.} Any prime divisor $p$ of $F_{n}$ where $n \geq 2$ is of the form $p = k\cdot 2^{n+2}+1$. \\ \\
\section{Elliptic Curve}
\textbf{Definition.} An elliptic curve $E/Q: \; y^{2}=x^{3}+ax+b, \; a,b\in \mathbb{Q}$ should has no repeated root(smooth), and together with $\infty$(projective) where $\Delta=-2^{4}(4a^{3}+27b^{2})\neq 0$. \\ \\
\textbf{Definition.} For $E/Q$, 
\begin{align}
E(Q) = \{(x,y) \; | \; x.y \in \mathbb{Q} \; and \; y^{2}=x^{3}+ax+b\} \cup \{\infty\}
\end{align}
is the set of Q-rational points of $E$. \\ \\
\textbf{Definition.} let. $P=(x_{1}, y_{1})$ and $Q=(x_{2},y_{2})$.
\begin{align}
&(1) \; if \; Q=(x_{2},y_{2})=(x_{1},-y_{1}), \; \; P+Q=\infty \\
&(2) \; if \; Q=P, \; \; P+Q=2P \; as: \\
&\quad \quad \textnormal{Find the tangent line which pass $P$ and find intersection of tangent line and $E$.} \\ 
&\quad \quad \textnormal{Just let $R=(x_{3},y_{3})$. Then $2P=(x_{3},-y_{3})$.} \\
&(3) \; if \; Q\neq P, \; \; \textnormal{Find the segment intersection of it and $E$.}
\end{align}
\newpage
\textbf{Theorem 3.1.} For $P_{1}, \; P_{2}, \; P_{3} \in \mathbb{Q}$,
\begin{align}
&(1) \; P_{1}+P_{2}\in E(Q) \\
&(2) \; P_{1}+P_{2}=P_{2}+P_{1} \\
&(3) \; P_{1}+(P_{2}+P_{3})=(P_{1}+P_{2})+P_{3}.
\end{align}
\textit{\textbf{Remark.}} $(E(Q), \; +)$ forms abelian qroup with identity $\infty$. \\ \\
\textbf{Theorem 3.2 Mordell-Weil.} Given $E/Q$, $\exists \infty$ly many $\mathbb{Q}$ sol. $P_{1},\ldots ,P_{n}$ s.t. $\forall P\in E(Q)$ is of the form $P=\sum^{m}_{j=1}n_{j}p_{j}$ for int. $n_{1},\ldots , n_{m}$. \\ \\
\textbf{Definition.} For prime $p$, $\mathbb{F}=\mathbb{Z}=\{0,\ldots ,p-1\}$ is a finite field order $p$. \\ \\
\textbf{Definition.} For $p\neq 2,3$, a prime, $E/\mathbb{F}_{p}$ is defined by $y^{2}=x^{3}+ax+b, \; a,b\in\mathbb{F}_{p}$ with
$\Delta=-2^{4}(4a^{3}+27b^{2})\not\equiv0 \; (mod \;p)$. Then,
\begin{align}
E(\mathbb{F})=\{(x,y) \; | \; x.y\in\mathbb{F}_{p} \; and \; y^{2}\equiv x^{3}+ax+b \; (mod \;p)\}
\end{align}
\textit{\textbf{Remark.}} If we count $\mathbb{Z}$ points fof $E$, we should consider $\infty$. For ex, 17 points $\Rightarrow$ total 18 points because of the existence of $\infty$. \\ \\
\textbf{Theorem 3.3 Hasse's bound.} $|\#E(\mathbb{F}_{p})-p-1|\leq2\sqrt{p}$ \\ \\
\textit{\textbf{Remark.}} Shimura-Taniyama-Weil Theorem and Birdu \& Swinnerton-Dyer Conjucture \\ \\
\section{Representation of Integers as Sums of Squares}
\textbf{Lemma.} If $m$ and $n$ are each the sum of two squares, then so is their product $mn$. \\ \\
\textbf{Theorem 4.1.} No prime $p$ of the form $4k+3$ is a sum of two squares. \\ \\
\textbf{Lemma Thue.} Let $p$ be a prime and gcd($a,p$) = 1. Then the congruence 
\begin{align}
ax\equiv y \; (mod \; p)
\end{align}
admits a sol. $x_{0}, y_{0}$, where
\begin{align}
0<|x_{0}|<\sqrt{p} \quad \quad and \quad \quad 0<|y_{0}|<\sqrt{p}
\end{align}
\textbf{Theorem 4.2 Fermat.} An odd prime $p$ is expressible as a sum of two squares iff $p\equiv 1 \; (mod \; 4)$. \\ \\
\textbf{Corollary.} Any prime $p$ of the form $4k+1$ can be represented uniquely (aside from the order of the summands) as a sum of two squares. \\ \\
\textbf{Theorem 4.3.} Let the positive int. $n$ be written as $n = N^{2}m$, where $m$ is squarefree. Then $n$ can be represented as the sum of two squares iff $m$ contains no prime factor of the form $4k+3$. \\ \\
\textbf{Corollary.} A positive int. $n$ is representable as the sum of two squares iff each of its prime factors of the form $4k+3$ occurs to an even power. \\ \\
\newpage
\textbf{Theorem 4.4.} A positive int. $n$ can be represented as the difference of two squares iff $n$ is not of the form $4k+2$. \\ \\
\textbf{Corollary.} An odd prime is the difference of two successive squares. \\ \\
\textbf{Theorem 4.5.} No positive int. of the form $4^{n}(8m+7)$ can be represented as the sum of three squares. Converse also holds.\\ \\
\textbf{Lemma 1 Euler.} If the int. $m$ and $n$ are each the sum of the four squares, then $mn$ is likewise so representable. \\ \\
\textbf{Lemma 2.} If $p$ is an odd prime, then the congruence 
\begin{align}
x^{2}+y^{2}+1\equiv 0 \; (mod \; p)
\end{align}
has a sol. $x_{0},y_{0}$ where $0\leq x_{0} \leq (p-1)/2$ and $0\leq y_{0} \leq (p-1)/2$ \\ \\
\textbf{Corollary.} Given an odd prime $p$, $\exists$ an int. $k<p$ s.t. $kp$ is the sum of four squares. \\ \\
\textbf{Theorem 13.6.} Any prime can be written as the sum of four squares. \\ \\
\textbf{Theorem 13.7 Lagrange.} Any positive int. can be written as the sum of four squares, some of which may be zero. \\ \\
\textit{\textbf{Remark.}} Waring's problem \& Easier one \\ \\
\section{Fibonacci Numbers}
\textit{\textbf{Remark.}} Fibonacci numbers grow rapidly! \\ \\
\textbf{Theorem 5.1.} For the Fibonacci sequence, gcd($u_{n},u_{n+1}$) = 1 for every natural $n$. \\ \\
\textit{\textbf{Fact.}} \; $3|u_{4n}, \; 5|u_{5n}, \; 7|u_{8n}$. \\ \\
\textbf{Lemma.} \; $u_{m+n} = u_{m-1}u_{n}+u_{m}u_{n+1}$. \\ \\
\textbf{Theorem 5.2.} For natural $m$ and $n$, $u_{mn}$ is divisible by $u_{m}$. \\ \\
\textbf{Lemma.} If $m=qn+r$, then gcd($u_{m},u_{n}$) = gcd($u_{r},u_{n}$). \\ \\
\textbf{Theorem 5.3.} The gcd of two Fibo. num. is again a Fibo. num; specifically,
\begin{align}
gcd(u_{m},u_{n})=u_{d} \quad \quad where \; d=(gcd(m,n)
\end{align}
\textbf{Corollary.} In the Fibo. sequence, $u_{m}|u_{n}$ iff $m|n$ for $n\geq m \geq 3$. \\ \\
\textbf{Corollary.} if $n>4$ is composite, then $u_{n}$ also. \\ \\
\textit{\textbf{Remark.}} If $u_{n}$ is prime, $n$ is odd prime or 4. \\ \\
\textbf{Lemma.} \; $u^{2}-u_{n+1}u_{n-1}=(-1)^{n-1}$ \\ \\
\textbf{Theorem 5.4.} Any positive int. $N$ can be expressed as a sum of distinct Fibo. num, no two of which are consecutive; that is,
\begin{align}
N=u_{k_{1}}+\cdots+u_{k_{r}}
\end{align}
where $k_{1}\geq 2$ and $k_{j+1}\geq k_{j}+2$ for $j=1,\ldots,r-1$. \\ \\
\textbf{Lemma 1.} \; $u_{3}+u_{5}+\cdots+u_{2s-1}=u_{2s}-1=u_{r}-1$. \\ \\
\textbf{Lemma 2.} \; $u_{2}+u_{4}+\cdots+u_{2s}=u_{2s-1}-1=u_{r}-1$. \\ \\
\textbf{Lemma.} \; $u_{n}=\cfrac{1}{\sqrt{5}}\left[\left(\cfrac{1+\sqrt{5}}{2}\right)^{n}-\left(\cfrac{1-\sqrt{5}}{2}\right)^{n}\right]$ \\ \\
\textbf{Theorem 5.5.} For a prime $p>5$, either $p|u_{p-1}$ or $p|u_{n+1}$, but not both. \\ \\
\textbf{Theorem 5.6.} Let $p\geq 7$ be a prime for which $p\equiv 2 \; (mod \; 5)$, or $p\equiv 4 \; (mod \; 5)$. If $2p-1$ is also prime, then $2p-1|u_{p}$. \\ \\
\section{Continued Fractions}
\textit{\textbf{Remark.}} Representation is not unique. \\ \\
\textbf{Theorem 6.1.} Any rational nu can be written as a finite simple continued fraction. \\ \\
\textbf{Definition.} \; $[a_{0};a_{1},\ldots,a_{n}]=[a_{0};a_{1},\ldots,a_{n-1}+1]$ \\ \\
\textbf{Definition.} For $[a_{0};a_{1},\ldots,a_{n}]$, by cutting off the expansion after the $k$th partial denomiator $a_{k}$ is called the $k$th convergent of the given continued fraction and denoted by $C_{k}$; in symbols,
\begin{align}
C_{k}=[a_{0};a_{1},\ldots,a_{k}] \quad 1\leq k \leq n
\end{align}
We let the zeroth convergent $C_{0}$ be equal to the number $a_{0}$. \\ \\
\textbf{Lemma.} $C_{k+1}=[a_{0};a_{1},\ldots,a_{k+1}]=[a_{0};a_{1},\ldots,a_{k}+\frac{1}{a_{k+1}}]$. \\ \\
\textbf{Definition.}
\begin{align}
&	p_{0}=a_{0} & q_{0}=1 \\
&	p_{1}=a_{1}a_{0}+1 & q_{1}=a_{1} \\
&	p_{k}=a_{k}p_{k-1}+p_{k-2} & q_{k}=a_{k}q_{k-1}+q_{k-2}
\end{align}
\textbf{Theorem 6.2.} $C_{k}=\cfrac{p_{k}}{q_{k}} \quad 0\leq k \leq n$. \\ \\
\textit{\textbf{Remark.}} It is convenient to define $p_{-2}=0,p_{-1}=1$ \quad and \quad $q_{-2}=1,q_{-1}=0$. \\ \\
\textbf{Theorem 6.3.} If $C_{k}$ is the $k$th convergent of the finite simple continued fraction, then 
\begin{align}
p_{k}q_{k-1}-q_{k}p_{k-1}=(-1)^{k-1}.
\end{align}
\textbf{Corollary.} For $1\leq k\leq n$, $p_{k}$ and $q_{k}$ are relatively prime. \\ \\
\textbf{Lemma.} If $q_{k}$ is the denominator of the $k$th convergent $C_{k}$ of the simple continued fraction, then $q_{k-1}\leq q_{k}$, with strict inequality when $k>1$. \\ \\
\textbf{Theorem 6.4.} $\forall$ natural n,
\begin{align}
C_{0}<C_{2}<\cdots<C_{2n}<C_{2n+1}<\cdots<C_{3}<C_{1}.
\end{align}
\textbf{Definition.} If $a_{0},\ldots$ is an infinite sequence of int, all positive except possibly $a_{0}$, then the infinite simple contunued fraction $[a_{0};a_{1},a_{2},\ldots]$ has the value
\begin{align}
\lim_{n\rightarrow\infty}[a_{0};a_{1},a_{2},\ldots,a_{n}]
\end{align}
\textit{\textbf{Remark.}} 
\begin{align}
\lim_{n\rightarrow\infty}\cfrac{u_{n+1}}{u_{n}}=\cfrac{1+\sqrt{5}}{2}
\end{align}
\textbf{Theorem 6.5.} The value of any infinite continued fraction is irrational. \\ \\
\textbf{Theorem 6.6.} Two distinct infinite continued fractions represents two distinct irrational numbers, i.e. representation is unique. \\ \\
\textit{\textbf{Remark.}} First let
\begin{align}
a_{k}=[x_{k}] \quad \quad x_{k+1}=\cfrac{1}{x_{k}-a_{k}}.
\end{align}
Then $x_{0}=[a_{0};a_{1},\ldots,a_{n},x_{n+1}]=C'_{n+1}=\cfrac{x_{n+1}p_{n}+p_{n-1}}{x_{n+1}q_{n}+q_{n-1}}$. \\ \\
Because of this,
\begin{align}
x_{0}-C_{n}=\cfrac{x_{n+1}p_{n}+p_{n-1}}{x_{n+1}q_{n}+q_{n-1}}-\cfrac{p_{n}}{q_{n}}=\cfrac{(-1)^{n}}{(x_{n+1}q_{n}+q_{n-1})q_{n}} \quad \Rightarrow \quad |x_{0}-C_{n}|<\cfrac{1}{q_{k}^{2}}.
\end{align}
\textbf{Theorem 6.7.} Every irrational has a unique representation as an infinite continued fraction, which obtained from the continued fraction algorithm described as (75). \\ \\
\textbf{Lemma.} Let $p_{n}/q_{n}$ be the $n$th convergents of the continued fraction representing the irrational number $x$. If $a$ and $b$ are int, with $1\leq b<q_{n+1}$, then
\begin{align}
|q_{n}x-p_{n}|\leq|bx-a|
\end{align}
\textbf{Theorem 6.8.} If $1\leq b\leq q_{n}$, the irrational $a/b$ satisfies
\begin{align}
\left|x-\cfrac{p_{n}}{q_{n}}\right|\leq \left|x-\cfrac{a}{b}\right|
\end{align}
\newpage
\textbf{Theorem 6.9.} Let $x$ be an arbitrary irrational. If the rational $a/b$, where $b\geq 1$ and gcd($a,b$) = 1, satisfies
\begin{align}
\left|x-\cfrac{a}{b}\right|<\cfrac{1}{2b^{2}},
\end{align}
then $a/b$ is one of the convergents $p_{n}/q_{n}$ in the continued fraction representation of $x$. \\ \\
\textit{\textbf{Remark.}} When deal with Pell's equation, we only consider positive sol. \\ \\
\textbf{Theorem 6.10.} If $p, \;q$ is a positive sol. of Pell's eq, then $p/q$ is a convergent of the continued fraction expansion of $\sqrt{d}$. \\ \\
\textbf{Theorem 6.11.} If $p, \;q$ is a convergent of the continued fraction expansion of $\sqrt{d}$, then there are a sol. of one of the eq.
\begin{align}
x^{2}-dy^{2}=k
\end{align}
where $|k|<1+2\sqrt{d}$. \\ \\
\textit{\textbf{Remark.}} All irrational took the periodic infinite sequence. \\ \\
\textit{\textbf{Remark.}}
\begin{align}
x_{0}=\sqrt{d} \quad and \quad x_{k+1}=\cfrac{1}{x_{k}-[x_{k}]} \quad \Rightarrow \quad x_{k+1}=\cfrac{1}{x_{k}-a_{k}}.
\end{align}
\textbf{Lemma.} Given the continued fraction expansion $\sqrt{d}=[a_{0};a_{1},a_{2},\ldots]$, define $s_{k}$ and $t_{k}$ recursively by the relations
\begin{align}
& s_{0}=0 \quad t_{0}=1 \\
& s_{k+1}=a_{k}t_{k}-s_{k} \quad t_{k+1}=\cfrac{d-s_{k+1}^{2}}{k} \quad k=\mathbb{Z}_{>0}
\end{align}
Then
\begin{align}
&(a) \; s_{k}, t_{k}\in \mathbb{Z}, \; t_{k}\neq 0 \\
&(b) \; t_{k}|(d-s_{k}^{2}) \\
&(c) \; x_{k}=(s_{k}+\sqrt{d})/t_{k}, \; k\geq 0.
\end{align}
\textbf{Theorem 6.12.} If $p_{k}/q_{k}$ are the convergents of the continued fraction expansion of $\sqrt{d}$ then 
\begin{align}
p_{k}^{2}=dq_{k}^{2}=(-1)^{k+1}t_{k+1} \quad where \; t_{k+1}>0 \quad k\in\mathbb{Z}_{>0}
\end{align}
\textbf{Corollary.} If $n$ is the length of the period of the expansion of $\sqrt{d}$, then
\begin{align}
t_{j}=1 \quad \Longleftrightarrow \quad n|j
\end{align}
\textbf{Theorem 6.13.} Let $p_{k}/q_{k}$ be the convergents of the continued fraction expansion of $\sqrt{d}$ and let $n$ be the length of the expansion.
\begin{align}
&(a) \; n=2k \Rightarrow \textnormal{All positive sol. of Pell's eq. are given by} \\
& \quad \quad \quad \quad x=p_{kn-1} \quad y=q_{kn-1} \\
&(b) \; n=2k+1 \Rightarrow \textnormal{All positive sol. of Pell's eq. are given by} \\
& \quad \quad \quad \quad x=p_{2kn-1} \quad y=q_{2kn-1} \quad \quad \quad k\in\mathbb{Z}_{>0}
\end{align}
\textbf{Theorem 6.14.} Let $x_{1}, y_{1}$ be the fundamental solution of Pell's eq. Then every pair of int. $x_{n}, y_{n}$ defined by the condition
\begin{align}
x_{n}+y_{n}\sqrt{d}=(x_{1}+y_{1}\sqrt{d})^{n} \quad \quad n\in\mathbb{N}
\end{align}
Also, every positive sol. of the eq. are determined as above. \\ \\
This is end. $\quad\blacksquare$
\end{document}
