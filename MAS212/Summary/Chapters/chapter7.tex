\documentclass[C:/LATEX/MAS212/Summary/MAS212.tex]{subfiles}
\usepackage{tikz}
\begin{document}
\section{Cyclic Subspaces and Annihilaters}

\dfn{$T$-Cyclic Subspaces}{
    $T:$ endo. on f.d.v.s. $V/F$. Take $\alpha\in V$. Then the $T$-cyclic subspace generated by $\alpha$ is denoted as $Z(\alpha\,;\,T):=\{g(T)\alpha\in V\;|\;g(x)\in F[x]\}$. 
    Just in case $Z(\alpha\,;\,T)=V$, we say $V$ is cyclically generated by $\alpha$ and $T$, and $\alpha$ is a cyclic vector for $T$.
}

\nt{
    $Z(\alpha\,;\,T)$ is always $T$-invariant. Also, $Z(\alpha\,;\,T)$ is very sensitive to choice of $\alpha$. 
    If $\alpha=0$, nothing no show. If $\alpha$ is a char. vec., then $T\alpha=c\alpha$, so $Z(\alpha\,;\,T)=\spans\{\alpha\}$, which implies 1-dimensional. Also note that converse holds.
}

\dfn{$T$-Annihilaters}{
    The $T$-annihilater, denoted as $M(\alpha\,;\,T):=\{g(x)\in F[x]\;|\;g(T)\alpha=0\}$.
}

\nt{
    Note that annihilator is just a special case of conductor, which takes $W=0$. We can also see that monic generator of annihilater divides minimal poly.
}

\thm{}{
    $T:$ endo. on f.d.v.s. $V/F$. $p_\alpha:$ $T$-annihilator of $\alpha$. Then
    \begin{enumerate}
        \item[i)] $\deg(p_\alpha)=\dim(Z(\alpha\,;\,T))$
        \item[ii)] If $\deg(p_\alpha)=k$, then $\{\alpha,T\alpha,\ldots,T^{k-1}\alpha\}$ forms a basis of $Z(\alpha\,;\,T)$
        \item[iii)] Let $U:=T|_{Z(\alpha\,;\,T)}:Z\mapsto Z$. Then min. poly. of $U$ is $p_\alpha(x)$.   
    \end{enumerate}
}

\pf{Proof}{
    Take $g(x)=p_\alpha q(x)+r(x)$ by Euclidean algorithm for $\deg(p_\alpha)=k$. Note that $(p_\alpha)=M(\alpha\,;\,T)$. Thus $p_\alpha q\in M(\alpha\,;\,T)\Rightarrow g(T)\alpha=p_\alpha(T)q(T)\alpha+r(T)\alpha=r(T)\alpha\Rightarrow Z(\alpha\,;\,T)=\spans\{\alpha,T\alpha,\ldots,T^{k-1}\alpha\}$.
    Thus $\dim(Z(\alpha\,;\,T))\leq k$.
    \clm[indep]{}{
        $\{\alpha,T\alpha,\ldots,T^{k-1}\alpha\}$ is a linearly independent set.
    }

    \pf{Proof}{
            Suppose not. Then there is nonzero coefficients satisfying $\sum c_iT^i\alpha=0$. Clearly $g(x)=\sum c_ix^i$ has $\deg<k$. But $p_\alpha$ is the nonzero poly. of min. deg. in $M(\alpha\,;\,T)$, while $g(x)\in M(\alpha\,;\,T)$ with degree less than $p_\alpha(x)$.
            This is contradiction, so this set is linearly independent.
        }

    Thus by Claim \ref{clm:indep}, $Z(\alpha\,;\,T)$ is $k$-dimensional with $\deg(p_\alpha)=k$. i) and ii) done.

    For iii), need to check $p_\alpha(U)=0$ and it really is poly. with min. deg. 

    An arbitrary element of $Z(\alpha\,;\,T)$ is of the form $g(T)\alpha$ for some $g(x)\in F[x]$. Thus $p_\alpha(U)g(T)\alpha=p_\alpha(T)g(T)\alpha=g(T)p_\alpha(T)\alpha=0$. Our first condition holds.
    Second condition is immediate from the minimality of the degree of $p_\alpha$ in $M(\alpha\,;\,T)$.
}

\mlemma{Companion Matrices}{
    $T:$ endo. on f.d.v.s. $V/F$. $W=Z(\alpha\,;\,U)\subset V$ where $U:=T|_{Z(\alpha\,;\,T)}$. Then w.r.t. the basis $\{\alpha,T\alpha,\ldots,T^{k-1}\alpha\}=\mfB$ of $Z$, 
    $[U]_\mfB=\begin{bmatrix}
        0 & & & -c_0 \\ 1 & \ddots & & -c_1 \\ & \ddots & \ddots & \vdots \\ & & 1 & -c_{k-1} 
    \end{bmatrix}$ where $p_\alpha=x^k+\sum_{i=0}^{k-1}c_ix^i$. This matrix is called companion matrix.
}

\pf{Proof}{
    $\mfB:=\{\alpha,T\alpha,\ldots,T^{k-1}\alpha\}=\{\alpha_1,\ldots,\alpha_k\}$. Then $U\alpha_1=T\alpha=\alpha_2$, $U\alpha_2=T^2\alpha=\alpha_3$, and so on, $U\alpha_{k-1}=T^{k-1}\alpha=\alpha_k$.
    By our supposition of $p_\alpha$, $p_\alpha(U)\alpha=U^k\alpha+\sum_{i=0}^{k-1}c_iU^i\alpha$. Thus we can derive companion matrix of above form.
}

\thm{}{
    $U$ has a cyclic vec. $\iff$ there is some ordered basis s.t. $U$ is represented by the companion mat. of the min. poly. for $U$.
}

\cor{}{
    If $A$ is the companion mat. of a monic poly. $p$, then $p$ is both min. and char. poly. of $A$.
}

\section{Cyclic Decompositions and the Rational Form}

\dfn{Complementary Subspaces}{
    $T:$ endo. on f.d.v.s. $V/F$. $W\subset V$ as $T$-inv. subspaces. If $\exists$ $T$-inv. subspace $W'\subset V$ s.t. $V=W\oplus W'$, then we say $W'$ is a complementary $T$-inv. subspaces of $W$.
}

\dfn{$T$-Admissible}{
    $T:$ endo. on f.d.v.s. $V/F$. A subspace is $T$-admissible if $W$ is $T$-inv. and $\exists f(x)\in F[x]\;\exists\beta\in V\;\exists\gamma\in W\s(f(T)\beta\in W\Rightarrow f(T)\beta=f(T)\gamma)$.
}

\mlemma[condition]{}{
    $T:$ endo. on f.d.v.s. $V/F$. Suppose $W$ is $T$-inv. If its complementary $T$-inv. subspace exists, then $W$ is $T$-admissible.
}

\pf{Proof}{
    $W$ is trivially $T$-inv. Suppose $f(T)\beta\in W$ for $(f(x)\in F[x])\wedge(\beta\in V)$. Since $V=W\oplus W'$, $\beta=\gamma+\gamma'$ for unique $(\gamma\in W)\wedge(\gamma'\in W')$.
    Then $f(T)\beta=f(T)\gamma+f(T)\gamma'$. Since $f(T)\beta$ and $f(T)\gamma$ are $T$-inv. and in $W$, $f(T)\gamma'$ should be in $W$.
    Independence of $W$ and $W'$ implies thus $f(T)\gamma'=0$. Thus $f(T)\beta=f(T)\gamma$, so $W$ is $T$-admissible.
}

\thm[cdt]{Cyclic Decomposition Theorem}{
    $T:$ endo. on f.d.v.s. $V/F$. Let $W_0\subset V$ be any proper $T$-admissible subspace. $\exists \alpha_1,\ldots,\alpha_r\in V\backslash\{0\}$ with respective $T$-annihilators $p_1,\ldots,p_r$ s.t.
    \begin{enumerate}
        \item[i)] $V=W_0\oplus(\bigoplus_{i=1}^rZ(\alpha_i\,;\,T))$
        \item[ii)] $p_k\;|\;p_{k-1}$
    \end{enumerate}
    Furthermore, the integer $r$ and $p_i$ are uniquely determined by i), ii), and the fact that no $\alpha_k$ is 0.
}

\pf{Proof}{
    We will divide our proof to 4 steps. During our proof, we intentionally denote $f(T)\beta$ as $f\beta$. 

    \textcolor{blue}{\textit{\textbf{Before:}}} Let $\beta\in V\backslash W$. Consider $S(\beta\,;\,W):=\{g(x)\in F[x]\;|\;g(T)\beta\in W\}$. Then $\exists$ monic poly. generator $f$ s.t. $f(T)\beta\in W$. By $T$-admissibility, $\exists\gamma\in W$ s.t. $f(T)\beta=f(T)\gamma$. Let $\alpha:=\beta-\gamma$, then $f(T)\alpha=0$.
    Since $\gamma\in W$, we can see that $S(\alpha\,;\,W)=S(\beta\,;\,W)$ and $f$ is also the $T$-conductor of $\alpha$ to $W$. Since $f(T)\alpha=0$, $f\in M(\alpha\,;\,T)$. Thus $(f)=S(\alpha\,;\,W)\subset M(\alpha\,;\,T)$. Conversely, if $g\in M(\alpha\,;\,T)$, $g(T)\alpha=0\in W$ so $M(\alpha\,;\,T)\subset S(\alpha\,;\,T)$. Thus $S(\alpha\,;\,W)=M(\alpha\,;\,T)$ and $f$ is also a $T$-annihilater. 

    \clm{}{
        $W\cap Z(\alpha\,;\,T)=0$.
    }

    \pf{Proof}{
        Suppose $g(T)\alpha\in W\cap Z(\alpha\,;\,T)$. Then $g\in S(\alpha\,;\,W)=M(\alpha\,;\,T)$ implies $g(T)\alpha=0$. Thus $W\cap Z(\alpha\,;\,T)=0$, so $W+Z(\alpha\,;\,T)\Rightarrow W\oplus Z(\alpha\,;\,T)$.
    }

    \textcolor{blue}{\textit{\textbf{Step 1:}}} Let's make following observation: Let $W\subset V$ be a proper $T$-inv. subspace. Then $\max_{\alpha\in V}S(\alpha\,;\,W)$ is obtained by some $\beta\in V$, so that $\deg(S(\beta\,;\,W))$ is maximized.

    For the above $\beta$, $W+Z(\beta\,;\,T)$ is $T$-inv. and strictly larger than $W$. Applying this observation to the given $W_0\subset V:$ $T$-inv. proper subspaces. Then we obtain $\beta_1\in V$ s.t. $\deg(S(\beta_1\,;\,T))$ is maximized among $\deg(S(\beta\,;\,W))$. Again, take $W_2=W_1+Z(\beta_2\,;\,T)$, which leads $W_0\subsetneq W_1\subsetneq\cdots\subsetneq W_r=V$.
    
    From this, we can derive at least $V=W_0+\sum_{i=1}^rZ(\beta_i\,;\,T)$. Know Let's say $(p_k):=S(\beta_k\,;\,W_{k-1})$ has the maximum deg. among the conductors.

    \textcolor{blue}{\textit{\textbf{Step 2:}}} Take $W_i$, $\beta_i$, $p_i$ $i\in[r]$ as above. Fix $1\leq k\leq r$ and let $\beta\in V$. Suppose $(f)=S(\beta\,;\,W_{k-1})$. Write $f\beta=\beta_0+\sum_{i=1}^{k-1}g_i\beta_i$ for some $g_i\in F[x]$, $\beta_i\in W_i$.

    \clm{}{
        $\beta_0=f\gamma_0$ for some $\gamma_0\in W_0$ and $f\,|\,g_i$.
    }

    \pf{Proof}{
        If $k=1$, it means $W_0$ is $T$-admissible, so nothing to proof. Thus suppose $k>1$. By the Euclidean algorithm, $g_i=fh_i+r_i$. We want to prove all $r_i=0$. Let $\gamma:=\beta-\sum_{i=1}^{k-1}h_i\beta_i$. Then $\beta-\gamma=\sum_{i=1}^{k-1}h_i\beta_i\subset W_{k-1}$. This leads $S(\gamma\,;\,W_{k-1})=S(\beta\,;\,W_{k-1})$. 
        Also, $f\gamma=f\beta-\sum_{i=1}^{k-1}fh_i\beta_i=f\beta-\sum_{i=1}^{k-1}(g_i-r_i)\beta_i=\beta_0+\sum_{i=1}^{k-1}g_i\beta_i-\sum_{i=1}^{k-1}g_i\beta_i+\sum_{i=1}^{k-1}r_i\beta_i$. Thus $f\gamma=\beta_0+\sum_{i=1}^{k-1}r_i\beta_i\s\cdots\s(1)$. Toward contradiction, some $r_j\neq0$ and say that $j$ is the largest between such numbers.

        $f\gamma=\beta_0+\sum_{i=1}^{k-1}r_i\beta_i$ for nonzero $r_i$. Clearly $\dim(r_i)<\dim(f)\s\cdots\s(2)$. Consider conductor $(p):=S(\gamma\,;\,W_{j-1})$. With $ W_{j-1}\subset W_{k-1}$, $S(\gamma\,;\,W_{j-1})\subset S(\gamma\,;\,W_{k-1})=(f)$. Thus $f\,|\,p$, i.e., $p=fq$ for some $q\in F[x]$. Applying $g$ to $(2)$ leads $p(\gamma)=g\beta_0+\sum_{i=1}^{j-1}gr_i\beta_i+gr_j\beta_j$ where $p(\gamma)\in W_{j-1}$, $g\beta_0\in W_0\subset W_{j-1}$, $gr_i\beta_i\in W_i\subset W_{j-1}$. 
        This eq. leads $gr_j\beta_j\in W_{j-1}$, and thus $\deg(gr_j)\geq\deg(S(\beta_j\,;\,W_{j-1}))=\deg(p_j)$ by definition, and $\deg(p_j)\geq\deg(S(\gamma\,;\,W_{j-1}))$ by mazimality condition of $\beta_j$, where $\deg(S(\gamma\,;\,W_{j-1}))=\deg(p)=\deg(fg)$. Consequently, $\deg(r_i)\geq\deg(f)$, which is contradiction. Thus all $r_i=0$, and all $f\,|\,g_i$, and $(1)$ says $f\gamma=\beta_0\in W_0$. Since $W_0$ is $T$-admissible, $\exists\gamma_0\in W_0$ s.t. $f\gamma=\beta_0=f\gamma_0$.
    }

    \textcolor{blue}{\textit{\textbf{Step 3:}}} Now we will find $\{\alpha_1,\ldots,\alpha_r\}$ in $V$ which satisfies i) and ii).

    Take $\{\beta_1,\ldots,\beta_r\}$ as \textcolor{blue}{\textit{\textbf{Step 1}}}. Fix $1\leq k\leq r$. Apply \textcolor{blue}{\textit{\textbf{Step 2}}} to the vec. $\beta=\beta_k$ and the $T$-conductor $f=p_k$. We obtain $p_k\beta_k=p_k\gamma_0+\sum_{i=1}^{k-1}p_kh_i\beta_i$ for $\gamma_0\in W_0$. Let $\alpha_k:=\beta_k-\gamma_0-\sum_{i=1}^{k-1}h_i\beta_i$. 
    Since $\beta_k-\alpha_k\in W_{k-1}$, $S(\alpha_k\,;\,W_{k-1})=S(\beta_k\,;\,W_{k-1})=(p_k)$, and since $p_k\alpha_k=0$, we have $W_{k-1}\cap Z(\alpha_k\,;\,T)=\{0\}$. Because each $\alpha_k$ satisfies this condition, $W_k=W_0\oplus(\bigoplus_{i=1}^kZ(\alpha_i\,;\,T))$ and that $p_k$ is the $T$-annihilater of $\alpha_k$.

    Since $p_i\alpha_i=0$ for each $i$, we have the trivial relation $p_k\alpha_k=0+p_1\alpha_1+\cdots+p_{k-1}\alpha_{k-1}$. Apply \textcolor{blue}{\textit{\textbf{Step 2}}} with $\beta_i$ replaced by $\alpha_i$ and with $\beta=\alpha_k$, we can conclude $p_k$ divides each $p_i$ with $i<k$.

    \textcolor{blue}{\textit{\textbf{Step 4:}}} We will show $r$ and each poly. $p_r$ are uniquely determined by the conditions.

    Take $\gamma_i$, $g_i$ $i\in[s]$ that satisfies conditions either. We will show $r=s$ and $p_i=g_i$.

    The poly. $g_1$ is determined as the $T$-conductor of $V$ into $W_0$. Let $S(V\,;\,W_0)$ be the collection of poly. $f$ s.t. $\forall\beta\in V\s(f\beta\in W_0)$, i.e., poly. $f$ s.t. $R(f(T))\subset W_0$. Then $S(V\,;\,W_0)$ is nonzero ideal. $g_1$ is the monic generator of this. Each $\beta\in V$ has the form $\beta=\beta_0+f_1\gamma_1+\cdots+f_s\gamma_s$ and so $g_1\beta=g_1\beta_0+\sum_{i=1}^sg_1f_i\gamma_i$. Since each $g_i$ divides $g_1$, 
    we have $g_1\gamma_i=0$ for all $i$ and $g_1\beta=g_1\beta_0\in W_0$. Thus $g_1\in S(V\,;\,W_0)$. Since $g_1$ is the monic poly. of least deg. which sends $\gamma_1$ into $W_0$, we see that $g_1$ is the monic poly. of least deg. in the ideal $S(V\,;\,W_0)$. By the same argu., $p_1$ also, so $p_1=g_1$. Now note three facts:
    \begin{enumerate}
        \item $fZ(\alpha\,;\,T)=Z(f\alpha\,;\,T)$
        \item If $V=\bigoplus_{i=1}^kV_i$, where each $V_i$ is $T$-inv., $fV=fV_1\oplus\cdots\oplus fV_k$.
        \item If $\alpha$ and $\gamma$ have the same $T$-annihilator, then $f\alpha$ and $f\gamma$ have the same $T$-annihilater and thus $\dim(Z(f\alpha\,;\,T))=\dim(Z(f\gamma\,;\,T))$.
    \end{enumerate}
    Now, proceed induction to show that $r=s$ and $p_i=g_i$. Suppose $r\geq2$. Then $\dim(W_0)+\dim(Z(\alpha_1\,;\,T))<\dim(V)$ Since $p_1=g_1$, we know $\dim(Z(\alpha_1\,;\,T))=\dim(Z(\gamma_1\,;\,T))$. Thus $\dim(W_0)+\dim(Z(\gamma_1\,;\,T))<\dim(V)$. Then
    \begin{align*}
        & p_2V=p_2W_0\oplus Z(p_2\alpha_1\,;\,T) \\
        & p_2V=p_2W_0\oplus Z(p_2\gamma_1\,;\,T)\oplus\cdots\oplus Z(p_2\gamma_s\,;\,T)
    \end{align*}
    satisfies our desire. Furthermore, we conclude that $p_2\gamma_2=0$ and $g_2$ divides $p_2$. The argument can be reversed to show that $p_2$ divides $g_2$. Thus $g_2=p_2$.
}

\cor{}{
    If, $W$ is $T$-admissible, it has complementary $T$-inv. subspace. So with \Cref{lem:condition}, if and only if condition holds.
}

\thm{}{
    $T:$ endo. There is $\alpha\in V$ s.t. $T$-annihilator of $\alpha$ is equal to min. poly.
}

\pf{Proof}{
    With $W_0=0$, apply cyclic decomposition. Take $\alpha=\alpha_1$. $T$-conductor fo $\alpha_1$ to $W_0$ is $T$-annihilater of $\alpha_1$, which is the min. poly.
}

\thm{}{
    If $T$ has cyclic vec., then char. poly. of $T$ is equal to min. poly. of $T$.
}

\thm{Generalized Cayley-Hamilton Theorem}{
    $T:$ endo. on f.d.v.s. $V/F$. $m$ be min. poly. and $p$ be char. poly. Then
    \begin{enumerate}
        \item[i)] $p\,|\,f$
        \item[ii)] $p$ and $f$ have the same prime factors except for multiplicities
        \item[iii)] If $p=f_1^{r_1}\cdots f_k^{r_k}$, then $f=f_1^{d_1}\cdots f_k^{d_k}$ where $d_i$ is the nullity of $f_i(T)^{r_i}$ divided by the deg. of $f_i$.
    \end{enumerate}
}

\pf{Proof}{
    $i):$ trivial from Cayley-Hamilton Theorem.

    $ii):$ Cyclic decompose with $W_0$ says $\exists\alpha_1\sim\alpha_r$ s.t. $V=\bigoplus_{i=1}^rZ(\alpha_i\,;\,T)$ with $m(x)=p_1(x)$ which is $T$-annihilater of $\alpha_1$. $p_i\,|\,p_{i-1}$. Take $T_i:=T|_{Z(\alpha_i\,;\,T)}$. Since $Z(\alpha_i\,;\,T)$ is a cyclic vec. space with cyclic vec. $\alpha_i$, $p_i$ is min. poly. for $T_i$ is also char. poly. of $T_i$.
    Thus char. poly. $f(x)=\prod_{i=1}^rp_i$ and any prime factor of $m(x)$ divides $f(x)$ by i) while if a prime factor divides $f$, it divides one of $p_i$. Thus $p_i\,|\,p_{i-1}\,|\,\cdots\,|\,p_1=m(x)$. Thus each prime factor of $f$ also divides $m(x)$. 

    $iii):$ Apply primary decomposition: $W_i=N(f_i(T)^{r_i})$. Take $T_i:=T|_{W_i}$. Then $f_i(x)^{r_i}$ is the min. poly. of $T_i$. Applying ii) to $T_i$ its min. poly. Thus char. poly. of $T_i$ is $f_i^{d_i}$ with $d_i\geq r_i$. Here, $\dim(W_i)$ is $d_i\cdot\deg(f_i)$. So $d_i=\frac{\dim(W_i)}{\deg(f_i)}=N(f_i(T)^{r_i})/\deg(f_i)$.
}

\cor{}{
    $T:$ nilpo. endo. on n-d.v.s. $V/F$. Then char. poly. of $T$ is $x^n$. 
}

\pf{Proof}{
    $T$ is nilpo. $\Rightarrow$ $\exists N$ s.t. $T^N=0$ $\Rightarrow$ min. poly. $m(x)\,|\,x^N$ $\Rightarrow$ $m(x)=x^r$. Thus $f(x)=x^n$.
}

\section{The Jordan Form}

\nt{
    How to find Jordan form?
}

\pf{Solution}{
    \textit{\textbf{Step 1:}} char. poly. $f(x)=\prod_{i=1}^k(x-c_i)^{d_i}$ for distinct $c_i$ and $m(x)=\prod_{i=1}^k(x-c_i)^{r_i}$ for $1\leq r_i\leq d_i$. Take $W_i=N((T-c_iI)^{r_i})$ as primary decomposition theorem. Then $V=\bigoplus_{i=1}^kW_i$. $T_i:=T|_{W_i}$ where $m_i(x)$ of $T_i$ is $(x-c_i)^{r_i}$.
    
    \textit{\textbf{Step 2:}} For each $W_i$, let $N_i:=(T_i-c_iI):W_i\rightarrow W_i$. Then $N_i$ is nilpotent operator on $W_i$. Note that $T_i=N_i+c_iI$. Consider each $W_i$ the cyclic decomposition of $W_i$ w.r.t. $N_i$. So, $W_i=\bigoplus_{k=1}^{s_i}Z(\alpha_k\,;\,N_i)$. Take $\beta_j=\{\alpha_j,N_i\alpha_j,\ldots,N_i^{k_j-1}\alpha_j\}$. Then
    \begin{equation}\label{eqn:123}
        [N_i|_{Z(\alpha_j\,;\,N_i)}]_{\beta_j}=\begin{bmatrix}
            0 &  &  & 0 \\ 1 & \ddots &  & \vdots \\ & \ddots & \ddots & \vdots \\ & & 1 & 0 
        \end{bmatrix}\qquad\Rightarrow\qquad [T_i|_{Z(\alpha_j\,;\,N_i)}]_{\beta_j}=\begin{bmatrix}
            c_u &  &  & 0 \\ 1 & \ddots &  & \vdots \\ & \ddots & \ddots & \vdots \\ & & 1 & c_i
        \end{bmatrix}.
    \end{equation}
    Take $\mfB^i=\cup\beta_j$ . Then
    \begin{equation*}
        [T_i|_{W_i}]_{\mfB^i}=\begin{bmatrix}
            \Box & & \\ & \ddots & \\ & & \Box
        \end{bmatrix}
    \end{equation*}
    where each box is of the form at \ref{eqn:123} R.H.S. Then finally take $B=\cup\mfB^i$. This leads what we call Jordan form, where each small blocks are elementary Jordan blocks.
}





\section{Computation of Invariant Factors}

\textit{\textbf{This Chapter is Intentionally Skipped at Lectures.}}

\section{Summary; Semi-Simple Operators}

\textit{\textbf{This Chapter is Intentionally Skipped at Lectures.}}

\end{document}